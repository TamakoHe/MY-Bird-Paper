\section{Introduction}
\newcommand{\textbi}[1]{\textit{\textbf{#1}}}
Avian biodiversity is an important factor 
in ecosystems and contributes to 
economic and culture development~\cite{bc1}. 
Currently, the rapid economic development leads to fragmentation and degradation of habitat and overhunting 
which eventually cause many bird species becoming endangered~\cite{bc1,bc2}. 

Monitoring birds is the basis of further protection.
Many relative organizations have lunched campaigns 
tracking quantities and behaviors. 
Xu Shi et al., investigated the migration of
\textbi{Clanga clanga} and found the difference of 
migration patterns between adult and juvenile specimens~\cite{bc3}.
Eric R. Gulson-Castillo et al., discussed the impacts 
of space weather on bird migration~\cite{bc4}.

Manual monitoring is time-intensive and requires professional knowledge.
Deep learning, however, can utilize computers' computing capability to learn the characteristics of birds, 
reaching a high accuracy.
In recent years, methods of birds monitoring and identification based on deep learning are 
thriving. For example, Stefan Kahl and Stefan Kahl  
proposed a deep neural network BirdNET which can 
identify bird species by sound in an average precision of 
0.791 for single-species recordings~\cite{sa1}.
Within the domain of sound identification, methods
such as Gaussian mixture model (GMM),  support vector machine (SVM)
were adopted~\cite{sa2,sa3}. For birds images identification, many state-of-art deep learning designs
were proposed. Potluri, H et al. applied ResNet (He et al., 2016)~\cite{resnet} on 
CUB-200-2011 dataset~\cite{dataset1} and reached an accuracy of 96.5\%~\cite{sa4}.
Liu et al. presented a novel feature concentration Transformer 
(TransIFC) to extract the semantic information in birds images~\cite{sa5}. 
This design was tested on NABirds dataset~\cite{dataset2} with an accuracy of 90.9\%.

These researches focus on identification birds' species with 
high-quality sounds or images. However, in real-time monitoring, 
detection of locations of birds with complex backgrounds is the 
fundamental task. Currently, most of birds images datasets 
are designed for fine-grained classification such as CUB-200-2011~\cite{dataset1}, NABirds~\cite{dataset2} and DongNiao International Birds 10000~\cite{dataset3}
rather than for object detection. Unfortunately, 
popular general-purposed object detection datasets MS COCO~\cite{datasetcoco} and Pascal VOC~\cite{datasetvoc} 
are lack of birds images photoed in field environment.
Yuki Kondo et al. built a dataset of birds images for small object detection~\cite{datasetmva}.
Using this dataset, Da Huo et al. developed a model 
combined with Swin Transformer and CenterNet~;
Hao-Yu Hou et al. introduced ensemble fusion techniques to reach 
an average precision of 77.6\% at an IoU threshold of 0.5~\cite{Swint,centernet,mva1,mva2}.
Although ~\cite{datasetmva} provides an alternative of MS COCO~\cite{datasetcoco} and Pascal VOC~\cite{datasetvoc},
this dataset does not contain many occluded, crowded and backlighting 
samples and its backgrounds are restricted to a few certain scenes.

In most industrial cases, the object detection models are based on Convolution neural network or CNN which extracts 
the images' features with a succession of convolution layers and generates proposals of instances with a detection head.
Fast-RCNN~\cite{fastrcnn}, for example, can utilize convolution network such as ResNet~\cite{resnet} and RoI(region of interests)
Pooling to extract features and detect objects in certain regions of images. YOLOv2~\cite{yolov2} modified the detection head 
to avoid flattening and fully-connected layers that weakened the spatial features in RoI.
In 2017, I. Guyon et al~\cite{transformer}, design a model that apply multi self-attention layers and residual connection(transformer)
which greatly improved feature extraction abilities for time-series data. It became the new benchmark for natural language processing
and other deep learning application. In the field of computer vision, 
transformer-based neural network models have increasingly challenged the dominance of convolutional neural networks (CNNs) 
in the field of computer vision. Vision transformers(ViT)~\cite{vit} and other related architecture such as Swin transformers~\cite{Swint}
can excel in tasks like image classification, object detection, and segmentation. In 2020, Carion et al~\cite{DETR} utilized 
the powerful features extraction abilities of transformers architecture to build an end to end objects detection model.
However, applying global awareness on each image led to slow convergence and long training time which hinder its use in time-sensitive object detection tasks.
% 编译速度好慢,改几个字就要等好久,还是latexmk增量编译
To ameliorate this issue, Detr with improved denoising anchor boxes for end-to-end object detection(DINO)~\cite{dino}.DINO leverages denoising training by injecting noisy samples into the object query inputs during training
helping the model converge faster, making it more stable and robust under complex scenes.
Yian Zhao et al., added a transformer-like multi-attention head on one of the output dimensions of ResNet~\cite{resnet} backbone,
optimizing the CNN-based objects detection architecture and achieve a better performance on real-time detection(RE-DETR)~\cite{RTDETR}. Although 
these designs improved the detection and segement of general categories objects, the accuracy of birds detection 
is very concerning. By training the RE-DETR on MS-COCO dataset with 70 epoches, we found that the mAP of bird is only (TODO:filled the number)
but the mAP of all categories is (TODO:filled the number). 
% Intro的初稿总算完成了,好开心,method也快了
% TODO:一些具体数字还要继续填入
% 记得PULL!












In light of these issues, we proposed a dataset 
containing birds images in field environment TH-Birds. 
The dataset is publicly available at \href{https://github.com/TamakoHe/TH-Birds}{https://github.com/TamakoHe/TH-Birds}. 
Sample images in the proposed dataset were all collected from various field environment including river, lake and mountainous region. Every bird in these images is annotated with its bounding box, instance segment along with notes
regarding the characteristics of every instances.  In addition, we proposed 
a new method of data augmentation simulating the trees' braches' occlusion 
and other complex scenarios. The FPN layer and the loss function of the 
RT-DETR~\cite{RTDETR} model were also redesigned for better performance in bird detection. 

Experimental results demonstrated that this model increased the mAP by 
1.3\% on the TH-bird and MS-COCO dataset. Performance on the occluded 
and blurry images were also significantly increased. The results indicated 
that the proposed dataset and model have the potential to provide a new spotlight
of wide bird detection and recognization. 
